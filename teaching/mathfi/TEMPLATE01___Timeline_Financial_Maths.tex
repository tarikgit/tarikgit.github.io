%%%
%%% Financial Maths - Secondary School Teaching
%%% Tarik Ocaktan - 26/9/2021 - Luxembourg
%%%

\documentclass[a4paper,landscape]{article}
\usepackage{tikz}
\usetikzlibrary{snakes}
\usepackage[official]{eurosym}

\begin{document}

\section*{Calcul de la valeur acquise ($A$)}


  \begin{tikzpicture}[snake=zigzag, line before snake = 5mm, line after snake = 5mm]
    % draw horizontal line
    \draw (0,0) -- (8,0);
    \draw[snake] (8,0) -- (10,0);
    \draw (10,0) -- (16,0);

    % draw vertical line
    \draw (14,-1) -- (14,-9);

    % draw lines spanning from each period to final period
    \draw (2, -1) -- (2, -8); \draw[-stealth] (2, -8) -- (14, -8);
    \draw (4, -1) -- (4, -7); \draw[-stealth] (4, -7) -- (14, -7);
    \draw (6, -1) -- (6, -6); \draw[-stealth] (6, -6) -- (14, -6);
    %\draw (8, -1) -- (8, -5); \draw[-stealth] (8, -5) -- (14, -5);
    \draw (10, -1) -- (10, -4); \draw[-stealth] (10, -4) -- (14, -4);
    \draw (12, -1) -- (12, -3); \draw[-stealth] (12, -3) -- (14, -3);

    % insert formulas into plot
    \node[draw] at (13, -2.5) {$\cdot (1+i)^1$};
    \node[draw] at (13, -3.5) {$\cdot (1+i)^2$};
    %\node[draw,align=above] at (13, -4.5) {$\cdot (1+i)^{n-3}$};
    \node[draw] at (13, -5.5) {$\cdot (1+i)^{n-3}$};
    \node[draw] at (13, -6.5) {$\cdot (1+i)^{n-2}$};
    \node[draw] at (13, -7.5) {$\cdot (1+i)^{n-1}$};

    % add the "+" sign
    \node[draw] at (15, -3) {$+$};
    \node[draw] at (15, -4) {$+$};
    \node[draw] at (15, -6) {$+$};
    \node[draw] at (15, -7) {$+$};
    \node[draw] at (15, -8) {$+$};

    % add the general formula
    \node[draw] at (14, -10) {$A = a \cdot \frac{(1+i)^n-1}{i}$};

    % draw vertical lines
    \foreach \x in {0,2,4,6,8,10,12,14}
      \draw (\x cm,3pt) -- (\x cm,-3pt);

    % draw nodes
    \draw (0,0) node[below=3pt] {$  $} node[above=3pt] {$  0 $};
    \draw (2,0) node[below=3pt] {$ a $\euro{}} node[above=3pt] {\small{fin pér.}$1$};
    \draw (4,0) node[below=3pt] {$ a $\euro{}} node[above=3pt] {\small{fin pér.}$2$};
    \draw (6,0) node[below=3pt] {$ a $\euro{}} node[above=3pt] {\small{fin pér.}$3$};
    \draw (8,0) node[below=3pt] {$ a $\euro{}} node[above=3pt] {\small{fin pér.}$\ldots$};
    \draw (10,0) node[below=3pt] {$ a $\euro{}} node[above=3pt] {\small{fin pér.}$n-2$};
    \draw (12,0) node[below=3pt] {$ a $\euro{}} node[above=3pt] {\small{fin pér.}$n-1$};
    \draw (14,0) node[below=3pt] {$ a $\euro{}} node[above=3pt] {\small{fin pér.}$n$};
  \end{tikzpicture}

\end{document}
